\documentclass{article}
\usepackage{xcolor}
\usepackage{amsmath}
\pagecolor{black}
\color{white}
\title{\vspace*{\fill} \Huge\begin{center}SAN P$\Sigma$DRO'S NOTEBOOK\end{center} \vspace*{\fill}}
\date{}
\begin{document}
\maketitle
\vspace*{\fill}
\begin{center}
    \huge Vol. 1: Diary and Mathematics
\end{center}
\vspace*{\fill}
\newpage
\vspace*{\fill}
\begin{center}
    \Huge FORMULAS AND THEOREMS
\end{center}
\vspace*{\fill}
\newpage
\begin{center}
    \Huge
    Triangular Numbers
\vspace{1em}

    $\Sigma$ = $\frac{n(n+1)}{2}$
\end{center} \vspace{1em}
\Large
Let it be known that the Triangular Numbers are like finding the \textbf{sum} of a specifc
problem or an equation. $\textbf{n}$ being the total objects given and divide by $\textbf{2}$
later in the equation. \\
\vspace{1em}

This is actually one of my favorite formulas in all of mathematics.
It's so simple and very interesting in many ways. Additionally, not only
the Triangular Numbers is one of my favorite formulas, but the formula:
\vspace{1em}
\begin{center}
    \Huge
    $e^{i\pi} + 1 =0$ 
\end{center} \vspace{1em}
this formula here is called \textbf{Euler's Identity} where in his constant, $e$ is raised
to the complex number, $i$, and pi, $\pi$. This formula part, $e^{i\pi}$,
is equal to $-1$ itself. If you add in $+1$, you would get the perfect number of $0$.
\vspace{1em}

Euler, one of my reasons to love mathematics and devote my life just like his works.
Without the works of Euler, all of this will not be made if it wasn't for 
him. Thank you, Mentor.

\newpage
\vspace*{\fill}
\begin{center}
    \Huge Random solving
\end{center}
\vspace*{\fill}
\newpage
% Random Solving Page
\newpage
\begin{enumerate}
    \item $64^{\frac{2}{3}} = 16$\\
    Solution:
    \vspace{1em}
    \begin{equation*}
        \sqrt[3]{64}^2 \rightarrow 4^2 \rightarrow \boxed{16}
    \end{equation*}
\end{enumerate}

\newpage % Creates a new page for the DIARY
\vspace*{\fill}
\begin{center}
    \Huge DIARY
\end{center}
\vspace*{\fill}




\end{document}